\section{IMO 1962 Q4}

En esta sección se va a detallar la solución al problema Q4
correspondiente al año 1962. Realizaremos la demostración
en lenguaje natural del problema y a su vez se presentará la
correspondiente formalización en Lean.

La formalización que aquí se presenta ha sido inspirada en
las que se proponen en ???? realizadas por Kevin Lacker y
Healther Macbeth.



\noindent
\framebox{\parbox{\textwidth}{
  \textbf{Problema 2 (1962--Q4)}. Resolver la ecuación
  \[\cos²(x)+\cos²(2x)+\cos²(3x)=1\].}}

\textbf{Solución:}

Previo a la resolución como tal del problema, se van a
introducir una serie de definiciones que nos servirán
para llevar una notación más compacta:

\begin{definicion}\label{problema}
  De aquí en adelante usaremos la etiqueta \textbf{problema}
  para referirnos a la expresión que queremos resolver,
  es decir,
  \begin{equation}\label{expprob}
    \cos²(x)+\cos²(2x)+\cos²(3x)=1.
  \end{equation}
\end{definicion}

La formalización en Lean de esta definición es:
\begin{leancode}
 def problema (x : ℝ) : Prop :=
 cos x ^ 2 + cos (2 * x) ^ 2 + cos (3 * x) ^ 2 = 1
\end{leancode}

\begin{definicion}\label{funaux}
  De aquí en adelante usaremos la etiqueta
  \textbf{funauxiliar} para referirnos a la siguiente
  expresión:
  \begin{equation*}
    \cos(x)·(\cos²(x)-\frac{1}{2})·\cos(3x).
  \end{equation*}
\end{definicion}

La formalización en Lean en sete caso sería:
\begin{leancode}
def funcion_auxiliar (x : ℝ) : ℝ :=
cos x * (cos x ^ 2 - 1/2) * cos (3 * x)
\end{leancode}

Una vez introducido estas dos definiciones, se va a proceder
a demostrar que resolver la ecuación\( \_ \)problema es
equivalente a resolver la ecuación obtenida al igualar a
cero la funauxiliar.

Esta demostración se detallará a continuación, pero previo a
ella se necesita de la demostración de un lema previo:

\begin{lema}[Igualdad]\label{igualdadlema}
  Para cualquier \(x\) perteneciente al conjunto de los
  números reales, se verifica la siguiente igualdad:
  \begin{equation}\label{lemaigualdad}
    \frac{\cos²(x)+\cos²(2x)+\cos²(3x)-1}{4}=\text{funauxiliar}(x).
  \end{equation}
\end{lema}
\begin{demostracion}
  En primer lugar, según la definición vista en \ref{funaux}, se
  tendría que la expresión (\ref{lemaigualdad}) se convierte en:
  \begin{equation}\label{igualdad}
    \frac{\cos²(x)+\cos²(2x)+\cos²(3x)-1}{4}=
    \cos(x)·(\cos²(x)-\frac{1}{2})·\cos(3x).
  \end{equation}
  Ahora bien, para llevar a cabo la demostración comenzaremos
  introduciendo las conocidas definiciones del coseno del
  ángulo doble y del coseno del ángulo triple, que son:
  \begin{align}
    & \cos(2x)=2\cos²(x)-\frac{1}{2}\label{cos2}\\
    & \cos(3x)=4\cos³(x)-3\cos(x)\label{cos3},
  \end{align}
  donde estas dos propiedades en Lean se formalizan con los
  dos siguientes lemas:
  \begin{itemize}
\item \mint{lean}|cos_two_mul : cos (2 * x) = 2 * cos x ^ 2 - 1|
  \index{\url{cos_two_mul}}
\item \mint{lean}|cos_three_mul : cos (3 * x) = 4 * cos x ^ 3 - 3 * cos x|
  \index{\url{cos_three_mul}}
\end{itemize}

A continuación, se van a desarrollar los dos términos de la
igualdad (\ref{igualdad}). Comencemos por el primer término
de los dos:
\begin{align}\label{term1}
  \frac{\cos²(x)+\cos²(2x)+\cos²(3x)-1}{4}&=\frac{\cos²(x)+(2\cos²(x)-1)²}{4}\\
  &+\frac{(4\cos³(x)-3\cos(x))²-1}{4},
\end{align}
donde simplemente se han introducido las definiciones (\ref{cos2})
y (\ref{cos3}).

Desarrollando los cuadrados que aparecen en (\ref{term1}) y
simplificando los términos se acaba teniendo que:
\begin{equation}\label{term11}
  \frac{\cos²(x)+\cos²(2x)+\cos²(3x)-1}{4}=
  \cos(x)(\frac{3}{2}\cos(x)+4\cos⁵(x)-5\cos³(x))
\end{equation}

Por otro lado, desarrollando de manera totalmente análoga el
segundo término de la igualdad (\ref{igualdad}), se tiene que:
\begin{align}
  \cos(x)·(\cos²(x)-\frac{1}{2})·\cos(3x)&\stackrel{(*)}{=}\cos(x)·
                                           (\cos²(x)-\frac{1}{2})·
                                           (4\cos³(x)-3\cos(x)
                                           )\label{term2}\\
                                         &=\cos(x)·
                                           (4\cos⁵(x)-5\cos³(x)
                                           +\frac{3}{2}\cos(x))
                                           \label{term21},
\end{align}
donde en (\ref{term2}) se ha hecho uso de las definiciones (\ref{cos2})
y (\ref{cos3}).

De manera que observando las expresiones (\ref{term11}) y (\ref{term21}),
se puede concluir que la igualdad planteada en (\ref{igualdad}) es cierta.
De esta forma, ya se tendría el lema demostrado.
\end{demostracion}

Veamos la formalización en Lean de este lema:
\begin{leancode}
lemma Igualdad {x : ℝ} :
(cos x ^ 2 + cos (2 * x) ^ 2 + cos (3 * x) ^ 2 - 1) / 4 = funauxiliar x :=
begin
  rw funauxiliar,
  rw real.cos_two_mul,
  rw cos_three_mul,
  ring,
end
\end{leancode}

Una vez se ha introducido este lema, se puede proceder a la demostración
del lema que ya se adelantó y que consiste en probar que resolver la
expresión \textbf{problema}, (\ref{problema}), es equivalente a resolver
la expresión obtenidad de igualar a cero \textbf{funauxiliar},
(\ref{funaux}). Veámoslo:

\begin{lema}[Equivalencia]
  Resolver la expresión (\ref{expprob}) es equivalente a resolver
  la expresión
  \begin{equation}\label{probaux}
    \text{funauxiliar}(x)=0.
  \end{equation}
\end{lema}
\begin{demostracion}
  Esta demostración se trata de una doble implicación, por ello,
  separaremos las implicaciones:

  \noindent
  \framebox{\longrightarrow}
  Comencemos suponiendo que se puede resolver la expresión
  \begin{equation}\label{h1}\tag{h1}
    \cos²(x)+\cos²(2x)+\cos²(3x)=1,
  \end{equation}
  entonces se tiene, sin más que pasar todos los términos a un mismo
  lado de la igualdad, que también se puede resolver la expresión:
  \begin{equation}\label{h11}
    \cos²(x)+\cos²(2x)+\cos²(3x)-1=0.
  \end{equation}

  Como se tiene que podemos resolver la igualdad anterior,
  (\ref{h11}), y que está igualad a cero. Se tiene que se puede resolver
  cualquier múltiplo o submúltiplo de ella; en este caso consideraremos
  la expresión dividida por cuatro.

  Entonces, se verifica que podemos resolver la expresión:
  \begin{equation}\label{h12}
    \frac{\cos²(x)+\cos²(2x)+\cos²(3x)-1}{4}=0.
  \end{equation}

  Para concluir esta implicación de la demostración, bastaría con
  aplicar el Lema \ref{igualdadlema}. De manera que se verificaría
  que se puede resolver el problema (\ref{probaux}).

  \noindent
  \framebox{\longleftarrow} A continuación, supongamos que se puede
  resolver el problema
  \begin{equation}\label{h2}\tag{h2}
    \text{funauxiliar}(x)=0.
  \end{equation}
  Queremos demostrar que resolver este problema es equivalente a
  resolver el problema planteado en (\ref{expprob}).

  En primer lugar, aplicamos el Lema \ref{igualdadlema} sobre la
  hipótesis (\ref{h2}). De manera que se obtiene que la hipótesis
  se transforma en la siguiente:
  \begin{equation}
    \frac{\cos²(x)+\cos²(2x)+\cos²(3x)-1}{4}=0.
  \end{equation}
  Cuando se tiene una división igualda a cero, existen dos
  posibilidades: o bien que el numerador sea cero o que lo sea
  el denominador. De esta forma, se verifica que:
  \begin{equation}
    (\cos²(x)+\cos²(2x)+\cos²(3x)-1)=0 \lor 4=0
  \end{equation}
  
\end{demostracion}

La formalización en Lean de este Lema es la siguiente:
\begin{leancode}
lemma Equivalencia {x : ℝ} : problema x ↔ funauxiliar x = 0 :=
begin
  split,
  {intro h1,
  rw problema at h1,
  rw ← Igualdad,
  rw div_eq_zero_iff,
  left,
  rw sub_eq_zero,
  exact h1,},
  {intro h2,
  rw problema,
  rw ← Igualdad at h2,
  rw div_eq_zero_iff at h2,
  norm_num at h2,
  rw sub_eq_zero at h2,
  exact h2,},
end
\end{leancode}
\chapter{Problemas de las IMO en Lean. Parte 2}

En este capítulo se presentará la solución a un problema de las Olimpiadas
Internacionales de Mátematicas (IMO) cuyas formalización en Lean no se ha
llevado a cabo con anterioridad.

Para la realización de esta formalización, se ha hecho uso de la propuesta de
solución en lenguaje natural del problema en cuestión que aparece en ???.

\noindent
\framebox{\parbox{\textwidth}{
    \textbf{Problema 1 (2001--Q6)}. Sean \(a\), \(b\), \(c\) y \(d\) cuatro
    números enteros y positivos tales que verifican que \(a\) es mayor
    estrictamente que \(b\), que a su vez \(b\) es mayor estrictamente que \(c\)
    y a su vez \(c\) es mayor estrictamente que \(d\). Además estos números
    enteros verifican la siguiente relación:
    \begin{equation*}
      ac+bd=(a+b-c+d)(-a+b+c+d).
    \end{equation*}
    Demostrar que entonces se tiene que el número \(ab+cd\) no es primo.
  }}
\section{IMO 2001 Q2}
En esta nueva sección, se va a detallar la resolución del problema
Q2 propuesto en el año 2001. Como en el resto de planteamientos
anteoriores que se han detallado en el capítulo, se comenzará
presentado la demostración en lenguaje natural del problema y
luego su correspondiente formalización en Lean.

La formalización en Lean de este problema se ha basado en la que
recientemente propuso Tian Chen y que se puede  encontrar en
\cite{TCC}.

\noindent
\framebox{\parbox{\textwidth}{
    \textbf{Problema 4 (2001--Q2)}. Consideremos \(a\), \(b\) y
    \(c\) tres números reales y positivos cualesquieras. Demostrar
    que
    \begin{equation*}
      \frac{a}{\sqrt{a²+8bc}}+\frac{b}{\sqrt{b²+8ca}}+
      \frac{c}{\sqrt{c²+8ab}}≥1.
    \end{equation*}
  }}

Para llevar a cabo la formalización en Lean de este problema, es
necesario la importación de la teoría
\href{https://github.com/leanprover-community/mathlib/blob/master/src/analysis/special_functions/pow.lean}
{analysis.special\_functions.pow} (sobre las potencias) y también
habilitar el \href{https://leanprover.github.io/reference/other_commands.html#namespaces}
{espacio de nombre} de los números reales (para la simplificación
de notación de algunas funciones).

En Lean todo lo detallado se formaliza como sigue:
\begin{leancode}
import analysis.special_functions.pow

open real
\end{leancode}

Además, se han de introducir las variables \(a\), \(b\) y \(c\) que
en este problema pertenecen al conjunto de los números reales (la
condición de que son positivos se introducirán como hipótesis).
En Lean esto se formaliza de la siguiente forma:
\begin{leancode}
variables {a b c : ℝ}
\end{leancode}

\subsection{Resultados auxiliares}
Para llevar a cabo la resolución de este problema, se van a utilizar
una serie de lemas y teoremas auxiliares.

El primer resultado auxiliar es el que se presenta a continuación:

\begin{lema}[suma\(\_\)pos]
  Sean \(a\), \(b\) y \(c\) tres números reales y positivos, esto es:
  \begin{align}
    &0<a, \label{haq2}\tag{ha}\\
    &0<b, \label{hbq2}\tag{hb}\\
    &0<c. \label{hcq2}\tag{hc}
  \end{align}
  Entonces, se tiene que la suma de \(a\) elevado a la cuarta, \(b\)
  elevado a la cuarta y \(c\) elevado a la cuarta es positiva.
  Simbólicamente:
  \begin{equation}
    0<a⁴+b⁴+c⁴.
  \end{equation}
\end{lema}
\begin{demostracion}
  La demostración de este lema es bastante sencilla, se comienza
  usando el resultado de que la potencia de cualquier número
  positivo es también positiva. Utilizando esto sobre las hipótesis
  (\ref{haq2}), (\ref{hbq2}) y (\ref{hcq2}), se tienen las tres
  siguientes consecuencias:
  \begin{align}
    &0<a⁴, \label{ha4q2}\tag{ha1}\\
    &0<b⁴, \label{hb4q2}\tag{hb1}\\
    &0<c⁴. \label{hc4q2}\tag{hc1}
  \end{align}

  Ya para finalizar, basta con usar dos veces el resultado que nos
  dice que la suma de dos números postivos es positiva. Esto supone
  que, si usamos este resultado sobre las hipótesis (\ref{ha4q2})
  y (\ref{hb4q2}) se tiene que
  \begin{equation}\label{sum1q2}\tag{sum1}
    0<a⁴+b⁴.
  \end{equation}
  Reiteramos este procedimiento ahora usando dicho resultado sobre
  las hipótesis (\ref{sum1q2}) y (\ref{hc4q2}) y se obtiene:
  \begin{equation}\tag{sum2}
    0<a⁴+b⁴+c⁴.
  \end{equation}

  De esta forma, ya se tendría el resultado deseado.
\end{demostracion}

La formalización en Lean de este lema es la siguiente:
\begin{leancode}
lemma suma_pos (ha : 0 < a) (hb : 0 < b) (hc : 0 < c) :
0 < a ^ 4 + b ^ 4 + c ^ 4 :=
begin
  have ha1: 0 < a ^ 4:=pow_pos ha 4,
  have hb1: 0 < b ^ 4:=pow_pos hb 4,
  have hc1: 0 < c ^ 4:=pow_pos hc 4,
  have sum1: 0< a ^ 4 + b ^ 4 := add_pos ha1 hb1,
  have sum2: 0< a ^ 4 + b ^ 4 + c ^ 4 := add_pos sum1 hc1,
  exact sum2,
end
\end{leancode}
\index{suma\_pos}

Para la formalización en Lean de este lema se han utilizado los
siguientes lemas auxiliares:
\begin{itemize}
\item \mint{lean}|pow_pos {a : R} (H : 0 < a) : ∀ (n : ℕ), 0 < a ^ n|
  \indLema{pow\_pos}
\item \mint{lean}|add_pos {a b : α} (h: 0 < a) (h': 0 < b), 0 < a + b|
  \indLema{add\_pos}
\end{itemize}

Además, las únicas dos tácticas que se han usado son
\tactica{exact}{exact} y
\tactica{have}{have}.


El segundo resultado auxiliar que se va a presentar es el siguiente:

Enunciado del lema

\begin{leancode}
  lemma cota (ha : 0 < a) (hb : 0 < b) (hc : 0 < c) :
  a ^ 4 / (a ^ 4 + b ^ 4 + c ^ 4) ≤
  a ^ 3 / sqrt ((a ^ 3) ^ 2 + 8 * b ^ 3 * c ^ 3) :=
begin
  have ha3: 0 < a ^ 3:= pow_pos ha 3,
  have ha32: 0 ≤  (a ^ 3) ^ 2:= pow_two_nonneg (a ^ 3),
  have hb8: 0 < 8 * b ^ 3 := mul_pos (bit0_pos zero_lt_four) (pow_pos hb 3),
  have hbc: 0 < 8 * b ^ 3 * c ^ 3 := mul_pos hb8 (pow_pos hc 3),
  have hdenom1: 0 < (a ^ 3) ^ 2 + 8 * b ^ 3 * c ^ 3 := add_pos_of_nonneg_of_pos ha32 hbc,
  have hsqrt: 0 < sqrt((a ^ 3) ^ 2 + 8 * b ^ 3 * c ^ 3) := sqrt_pos.mpr hdenom1,
  have hdenom2 := suma_pos ha hb hc,
  have hdenom3: 0 ≤ a ^ 4 + b ^ 4 + c ^ 4 := hdenom2.le,
  rw div_le_div_iff hdenom2 hsqrt,
  rw pow_succ',
  rw mul_assoc,
  apply mul_le_mul_of_nonneg_left _ ha3.le,
  rw ← pow_succ',
  apply le_of_pow_le_pow _ hdenom3 zero_lt_two,
  rw mul_pow,
  rw sq_sqrt hdenom1.le,
  rw ← sub_nonneg,
  have desarrollo: (a ^ 4 + b ^ 4 + c ^ 4) ^ 2 - a ^ 2 * ((a ^ 3) ^ 2 + 8 * b ^ 3 * c ^ 3)
      = 2 * (a ^ 2 * (b ^ 2 - c ^ 2)) ^ 2 + (b ^ 4 - c ^ 4) ^ 2 +
        (2 * (a ^ 2 * b * c - b ^ 2 * c ^ 2)) ^ 2 := by ring,
  rw desarrollo,
  have h1: 0 ≤ (a ^ 2 * (b ^ 2 - c ^ 2)) ^ 2 := pow_two_nonneg _,
  have h1': 0 ≤ 2 * (a ^ 2 * (b ^ 2 - c ^ 2)) ^ 2:= mul_nonneg zero_le_two h1,
  have h2: 0 ≤ (b ^ 4 - c ^ 4) ^ 2:= pow_two_nonneg _,
  have h3: 0 ≤ (2 * (a ^ 2 * b * c - b ^ 2 * c ^ 2)) ^ 2:= pow_two_nonneg _,
  have aux1:= add_nonneg h2 h3,
  have tesis:= add_nonneg h1' aux1,
  rw add_assoc,
  exact tesis,
end
\end{leancode}

Por último, el tercer resultado auxiliar para resolver el ejercicio es el que
se presenta a continuación:

\begin{leancode}
  theorem imo2001_q2_aux (ha : 0 < a) (hb : 0 < b) (hc : 0 < c) :
  1 ≤ a ^ 3 / sqrt ((a ^ 3) ^ 2 + 8 * b ^ 3 * c ^ 3) +
      b ^ 3 / sqrt ((b ^ 3) ^ 2 + 8 * c ^ 3 * a ^ 3) +
      c ^ 3 / sqrt ((c ^ 3) ^ 2 + 8 * a ^ 3 * b ^ 3) :=
begin
  have h1: b ^ 4 + c ^ 4 + a ^ 4 = a ^ 4 + b ^ 4 + c ^ 4,
  {rw add_comm,
   rw ← add_assoc,},
  have h2: c ^ 4 + a ^ 4 + b ^ 4 = a ^4 + b ^ 4 + c ^ 4,
  {rw add_assoc,
   rw add_comm,},
   have h3: a ^ 4 / (a ^ 4 + b ^ 4 + c ^ 4) + b ^ 4 / (b ^ 4 + c ^ 4 + a ^ 4) ≤  
    a ^ 3 / sqrt ((a ^ 3) ^ 2 + 8 * b ^ 3 * c ^ 3) + 
    b ^ 3 / sqrt ((b ^ 3) ^ 2 + 8 * c ^ 3 * a ^ 3):= 
    add_le_add (cota ha hb hc) (cota hb hc ha),
    have h4: a ^ 4 / (a ^ 4 + b ^ 4 + c ^ 4) + b ^ 4 / (b ^ 4 + c ^ 4 + a ^ 4) 
    + c ^ 4 / (c ^ 4 + a ^ 4 + b ^ 4) ≤  
    a ^ 3 / sqrt ((a ^ 3) ^ 2 + 8 * b ^ 3 * c ^ 3) + 
    b ^ 3 / sqrt ((b ^ 3) ^ 2 + 8 * c ^ 3 * a ^ 3) +
    c ^ 3 / sqrt ((c ^ 3) ^ 2 + 8 * a ^ 3 * b ^ 3) :=
    add_le_add h3 (cota hc ha hb),
    have igualdad: a ^ 4 / (a ^ 4 + b ^ 4 + c ^ 4) + b ^ 4 / (b ^ 4 + c ^ 4 + a ^ 4) 
    + c ^ 4 / (c ^ 4 + a ^ 4 + b ^ 4) = 1,
    {rw h1,
     rw h2,
     rw ← add_div,
     rw ← add_div,
     rw div_self,
     apply ne_of_gt,
     exact suma_pos ha hb hc,},
     rw igualdad at h4,
     exact h4,
end
\end{leancode}


\subsection{Conclusión del problema}
Una vez ya se han introducido y formalizado los tres resultados
axiliares, se procederá a la formalización en sí de la resolución
como tal del problema.
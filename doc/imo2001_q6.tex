\section{IMO 2001 Q6}

En esta nueva sección, se va a detallar la resolución del problema Q6
propuesto en el año 2001. Como en el resto de planteamientos anteoriores
que se han detallado en el capítulo anterior, se comenzará presentado la
demostración en lenguaje natural del problema y luego su correspondiente
formalización en Lean.

Para la realización de esta formalización, se ha hecho uso de la propuesta de
solución en lenguaje natural del problema en cuestión que aparece en ???.

\noindent
\framebox{\parbox{\textwidth}{
    \textbf{Problema 1 (2001--Q6)}. Sean \(a\), \(b\), \(c\) y \(d\) cuatro
    números enteros tales que \(a > b > c > d > 0\). Supongamos que
    \begin{equation*}
      ac+bd = (a+b-c+d)(-a+b+c+d).
    \end{equation*}
    Demostrar que \(ab+cd\) no es primo.
  }}


%%% Local Variables:
%%% mode: latex
%%% TeX-master: "IMO_en_Lean"
%%% End:
